\documentclass[article]{jss}

%% -- LaTeX packages and custom commands ---------------------------------------

%% recommended packages
\usepackage{thumbpdf,lmodern}

%% another package (only for this demo article)
\usepackage{framed}

%% new custom commands
\newcommand{\class}[1]{`\code{#1}'}
\newcommand{\fct}[1]{\code{#1()}}


%% -- Article metainformation (author, title, ...) -----------------------------

%% - \author{} with primary affiliation
%% - \Plainauthor{} without affiliations
%% - Separate authors by \And or \AND (in \author) or by comma (in \Plainauthor).
%% - \AND starts a new line, \And does not.
\author{Toby Dylan Hocking\\Northern Arizona University
   \And Guillaume Bourque\\McGill University}
\Plainauthor{Toby Dylan Hocking, Guillaume Bourque}

%% - \title{} in title case
%% - \Plaintitle{} without LaTeX markup (if any)
%% - \Shorttitle{} with LaTeX markup (if any), used as running title
\title{A disk-based functional pruning algorithm for optimal changepoint
  detection in genomic data} 

\Plaintitle{A disk-based functional pruning algorithm for optimal changepoint
  detection in large genomic data} 

%\Shorttitle{A Short Demo Article in \proglang{R}}
\Shorttitle{Disk-based optimal changepoint detection in large data}

%% - \Abstract{} almost as usual
\Abstract{ This article describes a new algorithm and \proglang{R}
  package for optimal changepoint detection in large genomic data
  sets. Previous packages have used in-memory implementations, which
  limit application to relatively small data sets. The proposed
  PeakSegPipeline package implements disk-based storage in order to
  compute optimal changepoint models for the large data sets which are
  widespread in genomics.}

%% - \Keywords{} with LaTeX markup, at least one required
%% - \Plainkeywords{} without LaTeX markup (if necessary)
%% - Should be comma-separated and in sentence case.
\Keywords{Dynamic programming, optimal changepoint detection, peak
  detection, genomic data, \proglang{R}} 

\Plainkeywords{Dynamic programming, optimal changepoint detection, peak
  detection, genomic data, R} 

%% - \Address{} of at least one author
%% - May contain multiple affiliations for each author
%%   (in extra lines, separated by \emph{and}\\).
%% - May contain multiple authors for the same affiliation
%%   (in the same first line, separated by comma).
\Address{
  Toby Dylan Hocking\\
  Northern Arizona University\\
  School of Informatics, Computing, and Cyber Systems\\
  Flagstaff, AZ, USA\\
  E-mail: \email{Toby.Hocking@R-project.org}\\
  %URL: \url{https://eeecon.uibk.ac.at/~zeileis/}
}

\begin{document}


%% -- Introduction -------------------------------------------------------------

%% - In principle "as usual".
%% - But should typically have some discussion of both _software_ and _methods_.
%% - Use \proglang{}, \pkg{}, and \code{} markup throughout the manuscript.
%% - If such markup is in (sub)section titles, a plain text version has to be
%%   added as well.
%% - All software mentioned should be properly \cite-d.
%% - All abbreviations should be introduced.
%% - Unless the expansions of abbreviations are proper names (like "Journal
%%   of Statistical Software" above) they should be in sentence case (like
%%   "generalized linear models" below).

\section[Introduction: optimal changepoint detection in R]{Introduction: optimal changepoint detection in \proglang{R}} \label{sec:intro}

The PeakSegOptimal package provides an in-memory implementation of the
PeakSeg model \citep{Hocking-constrained-changepoint-detection}.

%% -- Manuscript ---------------------------------------------------------------

%% - In principle "as usual" again.
%% - When using equations (e.g., {equation}, {eqnarray}, {align}, etc.
%%   avoid empty lines before and after the equation (which would signal a new
%%   paragraph.
%% - When describing longer chunks of code that are _not_ meant for execution
%%   (e.g., a function synopsis or list of arguments), the environment {Code}
%%   is recommended. Alternatively, a plain {verbatim} can also be used.
%%   (For executed code see the next section.)

\section{Models and software} \label{sec:models}


% \begin{leftbar}
% Note that around the \verb|{equation}| above there should be no spaces (avoided
% in the {\LaTeX} code by \verb|%| lines) so that ``normal'' spacing is used and
% not a new paragraph started.
% \end{leftbar}

% \proglang{R} provides a very flexible implementation of the general GLM
% framework in the function \fct{glm} \citep{Chambers+Hastie:1992} in the
% \pkg{stats} package. Its most important arguments are
% \begin{Code}
% glm(formula, data, subset, na.action, weights, offset,
%   family = gaussian, start = NULL, control = glm.control(...),
%   model = TRUE, y = TRUE, x = FALSE, ...)
% \end{Code}
% where \code{formula} plus \code{data} is the now standard way of specifying
% regression relationships in \proglang{R}/\proglang{S} introduced in
% \cite{Chambers+Hastie:1992}. The remaining arguments in the first line
% (\code{subset}, \code{na.action}, \code{weights}, and \code{offset}) are also
% standard  for setting up formula-based regression models in
% \proglang{R}/\proglang{S}. The arguments in the second line control aspects
% specific to GLMs while the arguments in the last line specify which components
% are returned in the fitted model object (of class \class{glm} which inherits
% from \class{lm}). For further arguments to \fct{glm} (including alternative
% specifications of starting values) see \code{?glm}. For estimating a Poisson
% model \code{family = poisson} has to be specified.

% \begin{leftbar}
% As the synopsis above is a code listing that is not meant to be executed,
% one can use either the dedicated \verb|{Code}| environment or a simple
% \verb|{verbatim}| environment for this. Again, spaces before and after should be
% avoided.

% Finally, there might be a reference to a \verb|{table}| such as
% Table~\ref{tab:overview}. Usually, these are placed at the top of the page
% (\verb|[t!]|), centered (\verb|\centering|), with a caption below the table,
% column headers and captions in sentence style, and if possible avoiding vertical
% lines.
% \end{leftbar}

% \begin{table}[t!]
% \centering
% \begin{tabular}{lllp{7.4cm}}
% \hline
% Type           & Distribution & Method   & Description \\ \hline
% GLM            & Poisson      & ML       & Poisson regression: classical GLM,
%                                            estimated by maximum likelihood (ML) \\
%                &              & Quasi    & ``Quasi-Poisson regression'':
%                                            same mean function, estimated by
%                                            quasi-ML (QML) or equivalently
%                                            generalized estimating equations (GEE),
%                                            inference adjustment via estimated
%                                            dispersion parameter \\
%                &              & Adjusted & ``Adjusted Poisson regression'':
%                                            same mean function, estimated by
%                                            QML/GEE, inference adjustment via
%                                            sandwich covariances\\
%                & NB           & ML       & NB regression: extended GLM,
%                                            estimated by ML including additional
%                                            shape parameter \\ \hline
% Zero-augmented & Poisson      & ML       & Zero-inflated Poisson (ZIP),
%                                            hurdle Poisson \\
%                & NB           & ML       & Zero-inflated NB (ZINB),
%                                            hurdle NB \\ \hline
% \end{tabular}
% \caption{\label{tab:overview} Overview of various count regression models. The
% table is usually placed at the top of the page (\texttt{[t!]}), centered
% (\texttt{centering}), has a caption below the table, column headers and captions
% are in sentence style, and if possible vertical lines should be avoided.}
% \end{table}


%% -- Illustrations ------------------------------------------------------------

%% - Virtually all JSS manuscripts list source code along with the generated
%%   output. The style files provide dedicated environments for this.
%% - In R, the environments {Sinput} and {Soutput} - as produced by Sweave() or
%%   or knitr using the render_sweave() hook - are used (without the need to
%%   load Sweave.sty).
%% - Equivalently, {CodeInput} and {CodeOutput} can be used.
%% - The code input should use "the usual" command prompt in the respective
%%   software system.
%% - For R code, the prompt "R> " should be used with "+  " as the
%%   continuation prompt.
%% - Comments within the code chunks should be avoided - these should be made
%%   within the regular LaTeX text.

\section{Illustrations} \label{sec:illustrations}

For one sample, MACS2 detected one peak, and our PeakSegFPOP algorithm
can be used to find the optimal boundaries of that peak
(Figure~\ref{fig:one-peak}).

For another sample, MACS2 detected five peaks, and our PeakSegFPOP
algorithm can be used to find a more likely model with three peaks
(Figure~\ref{fig:three-peaks}).

% \begin{leftbar}
% For code input and output, the style files provide dedicated environments.
% Either the ``agnostic'' \verb|{CodeInput}| and \verb|{CodeOutput}| can be used
% or, equivalently, the environments \verb|{Sinput}| and \verb|{Soutput}| as
% produced by \fct{Sweave} or \pkg{knitr} when using the \code{render_sweave()}
% hook. Please make sure that all code is properly spaced, e.g., using
% \code{y = a + b * x} and \emph{not} \code{y=a+b*x}. Moreover, code input should
% use ``the usual'' command prompt in the respective software system. For
% \proglang{R} code, the prompt \code{"R> "} should be used with \code{"+  "} as
% the continuation prompt. Generally, comments within the code chunks should be
% avoided -- and made in the regular {\LaTeX} text instead. Finally, empty lines
% before and after code input/output should be avoided (see above).
% \end{leftbar}

\begin{figure}[t!]
\centering
\includegraphics{jss-figure-more-likely-models-one-peak}
\caption{\label{fig:one-peak} Default peak model from the baseline
  MACS2 algorithm (left) and the optimal peak model computed using the
  proposed PeakSegFPOP algorithm (right).}
\end{figure}

\begin{figure}[t!]
\centering
\includegraphics{jss-figure-more-likely-models-three-peaks}
\caption{\label{fig:three-peaks} The peak model from the baseline
  MACS2 algorithm detected five peaks (left), and our PeakSegFPOP
  algorithm can be used to find a more likely model with three peaks
  (right).}
\end{figure}
 
\begin{figure}[t!]
\centering
\includegraphics{jss-figure-target-intervals-models}
\caption{\label{fig:target-intervals-models} The PeakSegFPOP algorithm
  was used to compute optimal models for all of the data sets in the
  benchmark. Data sizes range from $N=10^2$ to $10^7$ weighted data to
  segment (x-axis); disk usage (top panel) and computation time
  (bottom panel) are log-linear $O(N \log N)$.}
\end{figure}
 

% \begin{CodeChunk}
% \begin{CodeInput}
% R> m_pois <- glm(Days ~ (Eth + Sex + Age + Lrn)^2, data = quine,
% +    family = poisson)
% \end{CodeInput}
% \end{CodeChunk}
% %
% To account for potential overdispersion we also consider a negative binomial
% GLM.
% %
% \begin{CodeChunk}
% \begin{CodeInput}
% R> library("MASS")
% R> m_nbin <- glm.nb(Days ~ (Eth + Sex + Age + Lrn)^2, data = quine)
% \end{CodeInput}
% \end{CodeChunk}
% %
% In a comparison with the BIC the latter model is clearly preferred.
% %
% \begin{CodeChunk}
% \begin{CodeInput}
% R> BIC(m_pois, m_nbin)
% \end{CodeInput}
% \begin{CodeOutput}
%        df      BIC
% m_pois 18 2046.851
% m_nbin 19 1157.235
% \end{CodeOutput}
% \end{CodeChunk}
% %
% Hence, the full summary of that model is shown below.
% %
% \begin{CodeChunk}
% \begin{CodeInput}
% R> summary(m_nbin)
% \end{CodeInput}
% \begin{CodeOutput}
% Call:
% glm.nb(formula = Days ~ (Eth + Sex + Age + Lrn)^2, data = quine, 
%     init.theta = 1.60364105, link = log)

% Deviance Residuals: 
%     Min       1Q   Median       3Q      Max  
% -3.0857  -0.8306  -0.2620   0.4282   2.0898  

% Coefficients: (1 not defined because of singularities)
%             Estimate Std. Error z value Pr(>|z|)    
% (Intercept)  3.00155    0.33709   8.904  < 2e-16 ***
% EthN        -0.24591    0.39135  -0.628  0.52977    
% SexM        -0.77181    0.38021  -2.030  0.04236 *  
% AgeF1       -0.02546    0.41615  -0.061  0.95121    
% AgeF2       -0.54884    0.54393  -1.009  0.31296    
% AgeF3       -0.25735    0.40558  -0.635  0.52574    
% LrnSL        0.38919    0.48421   0.804  0.42153    
% EthN:SexM    0.36240    0.29430   1.231  0.21818    
% EthN:AgeF1  -0.70000    0.43646  -1.604  0.10876    
% EthN:AgeF2  -1.23283    0.42962  -2.870  0.00411 ** 
% EthN:AgeF3   0.04721    0.44883   0.105  0.91622    
% EthN:LrnSL   0.06847    0.34040   0.201  0.84059    
% SexM:AgeF1   0.02257    0.47360   0.048  0.96198    
% SexM:AgeF2   1.55330    0.51325   3.026  0.00247 ** 
% SexM:AgeF3   1.25227    0.45539   2.750  0.00596 ** 
% SexM:LrnSL   0.07187    0.40805   0.176  0.86019    
% AgeF1:LrnSL -0.43101    0.47948  -0.899  0.36870    
% AgeF2:LrnSL  0.52074    0.48567   1.072  0.28363    
% AgeF3:LrnSL       NA         NA      NA       NA    
% ---
% Signif. codes:  0 '***' 0.001 '**' 0.01 '*' 0.05 '.' 0.1 ' ' 1

% (Dispersion parameter for Negative Binomial(1.6036) family taken to be 1)

%     Null deviance: 235.23  on 145  degrees of freedom
% Residual deviance: 167.53  on 128  degrees of freedom
% AIC: 1100.5

% Number of Fisher Scoring iterations: 1


%               Theta:  1.604 
%           Std. Err.:  0.214 

%  2 x log-likelihood:  -1062.546 
% \end{CodeOutput}
% \end{CodeChunk}



%% -- Summary/conclusions/discussion -------------------------------------------

\section{Summary and discussion} \label{sec:summary}


%% -- Optional special unnumbered sections -------------------------------------

\section*{Computational details}

% \begin{leftbar}
% If necessary or useful, information about certain computational details
% such as version numbers, operating systems, or compilers could be included
% in an unnumbered section. Also, auxiliary packages (say, for visualizations,
% maps, tables, \dots) that are not cited in the main text can be credited here.
% \end{leftbar}

The results in this paper were obtained using
\proglang{R}~3.4.1 with the
\pkg{MASS}~7.3.47 package. \proglang{R} itself
and all packages used are available from the Comprehensive
\proglang{R} Archive Network (CRAN) at
\url{https://CRAN.R-project.org/}.


\section*{Acknowledgments}

% \begin{leftbar}
% All acknowledgments (note the AE spelling) should be collected in this
% unnumbered section before the references. It may contain the usual information
% about funding and feedback from colleagues/reviewers/etc. Furthermore,
% information such as relative contributions of the authors may be added here
% (if any).
% \end{leftbar}


%% -- Bibliography -------------------------------------------------------------
%% - References need to be provided in a .bib BibTeX database.
%% - All references should be made with \cite, \citet, \citep, \citealp etc.
%%   (and never hard-coded). See the FAQ for details.
%% - JSS-specific markup (\proglang, \pkg, \code) should be used in the .bib.
%% - Titles in the .bib should be in title case.
%% - DOIs should be included where available.

\bibliography{jss-refs}


% %% -- Appendix (if any) --------------------------------------------------------
% %% - After the bibliography with page break.
% %% - With proper section titles and _not_ just "Appendix".

% \newpage

% \begin{appendix}

% \section{More technical details} \label{app:technical}

% \begin{leftbar}
% Appendices can be included after the bibliography (with a page break). Each
% section within the appendix should have a proper section title (rather than
% just \emph{Appendix}).

% For more technical style details, please check out JSS's style FAQ at
% \url{https://www.jstatsoft.org/pages/view/style#frequently-asked-questions}
% which includes the following topics:
% \begin{itemize}
%   \item Title vs.\ sentence case.
%   \item Graphics formatting.
%   \item Naming conventions.
%   \item Turning JSS manuscripts into \proglang{R} package vignettes.
%   \item Trouble shooting.
%   \item Many other potentially helpful details\dots
% \end{itemize}
% \end{leftbar}


% \section[Using BibTeX]{Using \textsc{Bib}{\TeX}} \label{app:bibtex}

% \begin{leftbar}
% References need to be provided in a \textsc{Bib}{\TeX} file (\code{.bib}). All
% references should be made with \verb|\cite|, \verb|\citet|, \verb|\citep|,
% \verb|\citealp| etc.\ (and never hard-coded). This commands yield different
% formats of author-year citations and allow to include additional details (e.g.,
% pages, chapters, \dots) in brackets. In case you are not familiar with these
% commands see the JSS style FAQ for details.

% Cleaning up \textsc{Bib}{\TeX} files is a somewhat tedious task -- especially
% when acquiring the entries automatically from mixed online sources. However,
% it is important that informations are complete and presented in a consistent
% style to avoid confusions. JSS requires the following format.
% \begin{itemize}
%   \item JSS-specific markup (\verb|\proglang|, \verb|\pkg|, \verb|\code|) should
%     be used in the references.
%   \item Titles should be in title case.
%   \item Journal titles should not be abbreviated and in title case.
%   \item DOIs should be included where available.
%   \item Software should be properly cited as well. For \proglang{R} packages
%     \code{citation("pkgname")} typically provides a good starting point.
% \end{itemize}
% \end{leftbar}

% \end{appendix}

% %% -----------------------------------------------------------------------------


\end{document}
